\section{Peano's default VTK dumps}

Peano's default VTK plotters project Peano's tree-based grid onto an
unstructured mesh, as VTK does not support tree meshes.
The underlying mechanism is straightforward:

\begin{enumerate}
  \item If a vertex is touched/read the first time, it is dumped as vertex of
  the unstructured output mesh if it is not refined, i.e.~if no other vertex
  does exist at the same location on a finer mesh level.
  \item If a hanging vertex is created, we add a new vertex to the output dump
  if this hanging vertex has not been written before. The plotters typically
  maintain one big hash map to bookkeep which hanging vertices have already been
  written.
  \item If the tree traversal runs into an unrefined cell, this cell is made a
  cell of the unstructured output mesh.
\end{enumerate}


\noindent
Please note that most visualisation codes (such as Paraview) do interpolate
bi-/tri-linearly within the cells. 
If a cell's face is adjacent to cells on finer levels, hanging vertices on the
face exist.
If they do not hold the linearly interpolated value, you will discover
visualisation artifacts.

\begin{remark}
  For smooth output pictures of meshes with adaptivity, you have to 
  \begin{itemize}
    \item set the plotted properties on hanging vertices due to $d$-linear
    interpolation, and
    \item you have to add the plotter mapping after you have invoked the mapping
    that does initialise the hanging vertices.
  \end{itemize}
\end{remark}


\noindent
Please note that the dumped unstructured mesh is a non-conformal mesh. 
Algorithms such as isosurface identification thus might yield invalid
results---it depends on your visualisation algorithms.
